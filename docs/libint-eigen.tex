\documentclass[12pt]{article}

\usepackage[onlytext]{MinionPro}
\usepackage{microtype}

\setlength{\parindent}{0in}                                                     % No indentation for every paragraph

\usepackage[left=3cm, right=3cm]{geometry}		                                % Margins left and right
\usepackage{hyperref}                                                           % Clickable table of contents in PDFs
\usepackage{datetime}                                                           % Currenttime

\usepackage{physics}
\usepackage{listings}                                                           % Code inclusion
\usepackage{color}

\definecolor{codegreen}{rgb}{0,0.6,0}
\definecolor{codegray}{rgb}{0.5,0.5,0.5}
\definecolor{codepurple}{rgb}{0.58,0,0.82}
\definecolor{backcolour}{rgb}{0.95,0.95,0.92}

\lstdefinestyle{mystyle}{
    backgroundcolor=\color{backcolour},
    commentstyle=\color{codegreen},
    keywordstyle=\color{magenta},
    numberstyle=\tiny\color{codegray},
    stringstyle=\color{codepurple},
    basicstyle=\ttfamily,
    breakatwhitespace=false,
    breaklines=true,
    captionpos=b,
    keepspaces=true,
    numbers=left,
    numbersep=5pt,
    showspaces=false,
    showstringspaces=false,
    showtabs=false,
    tabsize=2
}

\lstset{style=mystyle}

\lstset{ %
    language=C++,
    style=mystyle
}

\title{The libint-eigen interface}
\author{Laurent Lemmens}
\date{\today \hspace{6pt} \currenttime}

\begin{document}

\maketitle

\begin{center}
\line(1,0){250}
\end{center}

\tableofcontents
\newpage



\section{Terminology}
    In the LibInt2 basis set context, there is some terminology that should be cleared up. Let's start from the beginning. A \textit{primitive Gaussian} is a function of the following mathematical form:
    \begin{equation}
        \varphi_{\zeta, l_x, l_y, l_z, \vb{R}}(x, y, z) = N (x - X)^{l_z} (y - Y)^{l_y} (z - Z)^{l_z} \exp(-\zeta r^2) \thinspace ,
    \end{equation}
    in which
    \begin{equation}
        r^2 = x^2 + y^2 + z^2 \thinspace .
    \end{equation}
    $l_x$, $l_y$ and $l_z$ are called the angular momenta of the primitive. The sum
    \begin{equation}
        l = l_x + l_y + l_z
    \end{equation}
    of the angular momenta determines the type of primitive: $l=0$ refers to an s-type, $l=1$ to a p-type, etc. (analogously to naming of the eigenfunctions of the hydrogen atom). The position vector $\vb{R}$ specifies the center of the primitive. \\

    Often, a linear combination (also known as a \textit{contraction}) of primitives is taken, leading to a \textit{contracted GTO} (cGTO). These are the functions that are used as basis functions. (Note that a single primitive can also be used as a basis function, in which case the linear combination is just one times that primitive.) \textit{Molecular orbitals} (MOs) are written as a linear combination of basis functions (cGTOs), which in turn serve as a one-electron basis to antisymmetrize into the many-electron basis of the \textit{Slater determinants} (SDs). \cite{jensen2007}


\section{Are the contracted basis functions normalized?}

    When constructing a \lstinline{libint2::BasisSet} object as in say

\begin{lstlisting}
libint2::BasisSet obs ("STO-3G", atoms);
\end{lstlisting}

    the corresponding file \lstinline{sto-3g.g94} is read (which is located at your \lstinline{LIBINT_DATA_PATH} environment variable), in which LibInt2 finds the exponents and contraction coefficients for the given basis for a given element. \\

    In \lstinline{libint2/basis.h}, we can see the following code (edited for brevity):

\begin{lstlisting}
static ... read_g94_basis_library(...){
    ...
    ref_shells[Z].push_back(
        libint2::Shell{...}
        )
    ...
}
\end{lstlisting}

    which calls a specific constructor of \lstinline{libint2::Shell}:

\begin{lstlisting}
Shell(...) {
    // embed normalization factors into contraction coefficients
    renorm();
}
\end{lstlisting}

    that makes sure that the CGTO is normalized by including the normalization factor in the contraction coefficients. So, \textbf{yes}, LibInt2 internally works with normalized basis functions.



% % % REFERENCES % % %

\newpage

\bibliographystyle{unsrt}                                                       % Bibliography in chronological order
\bibliography{/Users/laurentlemmens/Documents/Archief/Bibliotheek/research_bib.bib}

\end{document}
